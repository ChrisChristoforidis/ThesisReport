\chapter{Conclusions}
\label{conclusions}

Weir wrote in his original paper in 1979 "The presence of a human operator as the controller places additional constraints on the control situation. He is limited in his ability to sense feedback cues and in the rapidity and precision with which control actions can be made.". Designing a human controller for any motor related task is not as simple as plugging a controller and solving an optimization problem or placing poles in the stable region. 
\par
Multiple considerations must be made for the human as physiological system. What was made clear from the start of cybernetics research is that neuromscular dyanmics should be somehow accounted for. In \cite{hess2012modeling,schwab2013,moore2012human} the neuromuscular dynamics were modeled as a mass spring damper system with predermined coefficients. Considering all the information gathered from previous attempts in modelling the human rider in single track vehicles, it is evident that when designing a human controller the system's feedback loops should have a correspondence with the human's own sensory feedback systems. It is known that when riding a bicycle or a motorcycle visual, proprioceptive  and vestibular feedback work together to give our cortex an estimation of the state of the vehicle we are controlling. In  \cite{hess2012modeling,schwab2013,moore2012human} it is made clear that proprioceptive feedback is linked with steer angle and steer rate feedback, which makes sense since the muscle spindles in our hands that are touching the handlebars should give us information on its perceived location and motion. As far as our perception of roll is concerned the vestibular and visual system  are involved. Hess et al.\cite{hess2012modeling} attributed roll rate feedback to the vestibular and the roll angle feedback to visual. Finally for the path following task, heading and lateral deviation changes are perceived  through our visual system (see Fig. \ref{fig:figure8}).

Another major consideration with human control of single track vehicles is the fact that the task is not entirely compensatory. Especially in the path following component preview plays a very significant role. This was explored extensively by Sharp \cite{sharp2007motorcycle,sharp2006optimal,sharp2008stability,sharp2007optimal,thommyppillai2009advances}. Another consideration is the human's ability to learn through the creation of internal models. It is a prelevant theory in motor control research that humans, for every motor related task, have an internal model of each process that estimates the state of the system. The internal model estimation is then fused with the delayed senosory feedback pathways resulting in more accurate and faster estimations of the state \cite{wolpert1995internal}. What makes it even more complicated is the fact that this internal model is dynamically changing. It changes as the human's experience with the task improves. Prem and Good \cite{prem1984rider} in their studies explored the effects of riding experience and found that lean body motions are less and less prevalent the more experienced the rider becomes, so the human operates on pure steer control. Edelmann et al.\cite{edelmann2015} incorporate an anticipatory feedforward level to their control architecture in order to counter the effects of the inherently delayed compensatory feedback control level.

The rider models of table \ref{table2} all fall under the umbrella of optimal control. These authors made the assumption that humans when trying to control single track vehicles operate as constrained optimal controllers, which is not that far fetched to believe considering past research on motor control \cite{wolpert2007probabilistic}. In \cite{chu2018modelling,massaro2012virtual},  the authors went a step further by  introducing the concept of MPC, which again comes back to the way our eternal shifting brain functions operate. One would not expect the brain to compute the optimal solution at time zero and then operate under the assumption that its initial computation is 100\% correct. A more reasonable approach would be a process in which our brains adapt and respond by computing optimal scenarios throughout the execution of the task. This is exactly what the MPC models are trying to simulate.

There is no "best" approach when considering the fact that a lot of models are trying to tackle different control tasks. A controller looking at pure stabilization will look different than one that tries to simulate the whole riding behavior from path following to roll stability. What makes the single track vehicle problem unique is their inherent instability at speeds outside the stable region. Design and validation of stabilization controller should be first priority.  While it has not been systematically proven the preferred method of control for the majority of the literature explored seems to be torque control (see tables \ref{table1}, \ref{table2}, \ref{table3}).Also worth noting is the fact that lean torque is often deemed insignificant especially in controllers where roll stabilization is the controlling task \cite{katayama1988control,van1970influence,hess2012modeling,schwab2013}. The importance of proprioceptive feedback in these must not be understated.

In conclusion complex controllers that tackle multiple tasks in multiple nested control loops are very hard to be validated with experiments so they will not be part of the focus of my research going forward. My aim wil be to implement a rider model similar to the Schwab and de Lange model \cite{schwab2013} that can be indirectly validated, throught system identification of its parameters in open road riding experiments.

