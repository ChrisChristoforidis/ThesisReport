\chapter*{Abstract}\label{chap:abstract}
\setheader{Abstract}

Bicycle riding is a fundamental part of everyday transportation in many countries around the world. Ever since the development of the safety bicycle, the bicycle remains  one of the most prominent means of transport. Despite cycling’s prominence in every day life for almost two centuries we still do not fully understand how the rider controls the bike in a more systematic way.

The purpose of this literature study is to explore and evaluate models of the single track vehicle rider found in the researched bibliography. Firstly a brief analysis of bicycle stability and its dynamics is presented as explored by Meijaard et al.\cite{meijaard2007linearized}. In Chapter \ref{results}models which were classified on their control theory approach are presented. In section \ref{ch:classical}, models using the classical control approach are described. Following the steps of early cybernetics research in which airplane pilot modelling was pioneered by McRuer\cite{mcruer1959human,mcruer1967manual,mcruer1967manual2}, a plethora of authors attempted adapting McRuers crossover model in order to model the rider of a seemingly much more complex task, motorcycle and bicycle riding. However some argued that such an approach will not work since cycling is not just a compensatory task. These spawned a new wave of research focusing in optimal control which is described in section \ref{ch:optimal}.  This approach has its roots in early motor control research in which the human brain is believed to work as a constrained optimal controller. In the final section of Chapter \ref{results}, all models that do not fall under the two approaches described above are reviewed. These include fuzzy logic controllers which have the advantage of incorporating heuristic  findings that are impossible to formulate using systematic mathematical approaches. Also intermittent control is briefly explored as a solution. 

Based on this literature review the graduation project will try to model the human controller for bicycles, while focusing  on the roll stabilization task. Open road experiments will be conducted in effort to estimate the controller’s parameters and indirectly validate it with real cycling data.



\begin{flushright}
{\makeatletter\itshape
    \@author \\
    Delft, April 2018
\makeatother}
\end{flushright}
