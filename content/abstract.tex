\chapter*{Abstract}\label{chap:abstract}
\setheader{Abstract}

 Experimental data were obtained from riding a steer-by-wire bicycle on the open road  while perturbing balance with impulsive forces at the seat post (lateral pertubations) as well as perturbing balance with impulsive torques at the steering assembly (steering pertubations). The experiments were conducted at 2.6–5.6 m/s covering both the stable and the unstable forward speed range. The lateral pertubation  dataset was acquired for two experimental conditions; normal cycling and reduced force feedback cycling. In an effort to investigate the effect of torque feedback  three metrics are defined to assess both steering and balancing performance for the two conditions. Results failed to indicate any statistically significant difference between experimental conditions. Bicycle and rider  mechanics have been modeled using the Whipple bicycle model extended with the rider inertia. A rider control model is developed that incorporates all of human's sensory pathways and includes a strategy to compensate for sensory dead time. The identified rider control parameters, stabilize the system and mimic realistic rider control behavior. From the results the importance of the torque feedback pathway is strongly indicated. Finally for the steering pertubations the rider control model is modified to account for the cocontraction mechanism. The model  manages to apprixamte the rider measured response  and simultaneously captures the significance of the intrinsic response. A high level of intersubject variability is exhibited. The hypothesis that this variability is in fact due to the modulation of admittance in the shoulder joint is strongly suggested.


\begin{flushright}
{\makeatletter\itshape
    \@author \\
    Delft, November 2019
\makeatother}
\end{flushright}
