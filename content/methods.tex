\chapter{Methods}
\label{methods}


In order to answer the research question posed, Google scholar was  used as the main tool of literature research. It proved to be the search engine with the most relevant and extensive literature.


In order to understand how a rider can be modelled as a controller, an understanding of single track vehicle dynamics is paramount. For that reason the first step was to seek literature that extensively describes the equations of motion of single track vehicles and analyses the unique problem arising from their inherent instability in low speeds. Search terms: “bicycle”, “dynamics”, “stability” , “motorcycle”, “equations + motion”, “eigenvalue + analysis”, “linear” , “non-linear”  were used.


\par
The next step was to find any published reviews on the subject of rider control for bicycles and motorcycles. The reviews of Kooijman et al.\cite{kooijman2013review}, Popov et al.\cite{popov2010review} and the bibliography review from Moore’s PHD thesis \cite{moore2012human} proved the most substantiated and extensive on the subject. A distinction was made between articles that were simply looking at bicycle and motorcycle control in general (ex. robot bicycles) and those that specifically tried to model the human controller. The latter  were more thoroughly visited. 

\par
In order to fill the gaps of more recent literature, an additional search on google scholar was conducted. The search was done by looking at papers involving keywords such as “rider control” , “rider identification” , “controller”, “optimal”, “McRuer”, “cybernetics”. For bicycle control results were filtered by looking at articles that directly cited the work of Meijaard et al. \cite{meijaard2007linearized} in bicycle dynamics, while for motorcycles results were filtered by looking at direct citations of Sharp’s work in motorcycle dynamics \cite{sharp1971stability}. These two articles involve the most prominent versions of linearized equations of motion for single track vehicle, so they are bound to be mentioned in any modern take on rider modelling.

Finally a search outside the scope of the problem was done looking at how the human controller is modelled in other motor related tasks. Search keywords included : “sensory-motor + integration” , “human + optimal + controller”, “Wolpert”, “Inverse + model”, “Internal + Model”, “feed-forward + control”.

All the relevant articles were stored in a database. If the article involved some form of control that the authour found unfamiliar, further research on the subject was conducted. These included fuzzy logic control, intermittent control and  model predictive control. After reading into the abstacts and conclusions of each article, an initial assesment was made on the relevancy of the work. Depending on that evaluation, a more extensive reading of the methods of some of the articles was done.