\chapter{Discussion}
 
In an effort to answer the research question regarding what is the importance of the torque feedback pathway in the balance task of cycling, a set of experiments were conducted using the experimental steer-by-wire platform that allowed us to create a unique steering configuration in which the steering assembly works like a decoupled master to the fork ("haptics off"). From the initial analysis of the raw measurements and the non-parametric impulse responses, results  suggested  that the effects of torque feedback are minimal to non-existent for the roll stabilization task. Neither performance (see figure \ref{fig:BoxPlots} (a)) or steering effort is affected by the removal of  steering torque feedback. Balance performance among conditions is comparatively consistent  (see figure \ref{fig:BoxPlots} (a)). However, in the unstable speed region the variance and the number of outliers are higher. For steering effort the null hypothesis that the $\mathit{PSC}_x$ metric came from independent random samples with equal means and equal variances failed to be rejected for all speed levels. This does not undoubtedly prove that the samples came from the same population, howbeit it gives a strong indication towards that fact. Additionally, for the "haptics  on"  the steering response was delayed (\(\approx 18\; \si{ms}\) see figure \ref{fig:BoxPlots} (b)) in comparison to the "haptics off". This is attributed to the fact that the handlebars are more inert due to the additional steering feedback, so the human response is "lagged" by the additional forcing dynamics.
 
From the analysis done on the gray box rider control model of \cref{hapticFB}, it is found that  the impact on the steering response from the removal of the torque feedback pathway is quite significant; \ensuremath{\mathit{VAF}_\delta} drops by at least 15\% for all models tested. Even in the variable delay model, torque feedback is potent enough to compensate for the delays and achieve over 90 \% fit in steering response for the lower forward speeds (see \cref{tb:variable}). This could be, because torque inherently includes acceleration information and can give  the rider a preview of how the rest of the state is going to evolve. However, in the haptics off condition, where the torque feedback is not physiologically severed but lessened due to the changed steering dynamics, major  degradation in fitting performance was only noted for the model with uncompensated time delays. In this case torque feedback is still proportional to the steering acceleration so the "preview" information is not completely lost. However the rider receives no input for the effect of the disturbance because the torque that would naturally transfer through the front wheel contact point is filtered by the decoupled fork-handlebar connection.  Despite all that, in the VDROP model, \ensuremath{VAF_\delta} did not drop more than 8\% and \ensuremath{VAF_\phi}, \ensuremath{VAF_\psi} showed an even more insignificant drop, which could explain why no difference was found between steering configurations in the IRFs. 
 
The  prediction algorithm used in the VDROP model  manages to utilize the Smith  principle (see \cref{fig:smith}) to enhance the optimal  predictions of the adapted discrete optimal predictor (see \cref{fig:VDROP_line}) with  information of the effect of the disturbance on the state through reafferent pathways and simultaneously compensate for internal model inaccuracies. In \cref{fig:predictor_compare2} it is visible that VDROP manages to achieve comparable performance to the perfect internal model case while adapted DOP fails. 
 
As future work the VDROP model can be extended to account for the fact that sensory feedback pathways are inherently noisy. However, incorporating noise,  without a way to compensate for it, is not realistic as has been shown by motor control research \cite{wolpert1995internal}. The accepted way in literature is to optimally fuse the output of an internal model with the the measurements through a Kalman filter. 
 
Finally, for the second set of experiments in which the participants are tasked with balancing under steering impulse perturbations, a similar analysis is conducted. Much greater intrasubject and intersubject variability is noted ( see \cref{fig:FIT_FIR,fig:IRF_all}). An effort to fit the same model as the one used for the lateral perturbations is made but the effect of passive rider dynamics on the response is so significant that it results in low \ensuremath{\mathit{VAF}} even for the ideal zero delay case. For this reason the model is enhanced by incorporating a second controller that modulates the intrinsic stiffness and damping of the shoulder joint. The updated model captures the significance of the cocontraction mechanism achieving a good level of fit (see \cref{fig:FIT_model}). An increasing reliance on intrinsic steering stiffness as speed increases is noted (see \cref{fig:gain_plots_steer}). On the other hand, the intrinsic modulation of steering damping is lessened as speed increases, in favor of the reflexive pathway (see \cref{fig:gain_plots_steer}). However, the assumption that for constant speeds the system is time invariant does not seem to hold ground as noted by the irregular effect of cocontraction in the later stages of the rider  response (see \cref{fig:param_input}). If further conclusions are to derived a more thorough analysis of the model needs to be conducted by doing a proper parameter sensitivity analysis similar to how it was done in \cref{ch:4}.
 
Although a direct comparison between the rider models of \cref{hapticFB} and \cref{ch:4} is not applicable as the former incorporates heading as a feedback gain while the latter does not, some insights can be still be gained from the comparison of the identification results. First of all the the "steer into the fall" mechanism is prominent in both as can be seen by the negative proportional and derivative roll angle gains. However, while in the lateral perturbations modulation of steering stiffness and damping are of little to no importance the opposite is the case for the steering perturbation control model. In the zero delay model, gain \ensuremath{K_\delta} was successfully deprecated without  significant drop in \ensuremath{\mathit{VAF}}. When delays and the prediction algorithm was introduced \ensuremath{K_{\dot{\delta}}} exhibited similar effect when removed. On the other hand in the controller of \cref{ch:4} these two gains are so important that even in the case with zero delays the model was unable to match the measured response because of the lag induced by the neuromoscular dynamics. For this reason the intrinsic controller was added that enhances the effects of steering stiffness and damping through the cocontraction mechanism. Interesting is the difference between \ensuremath{K_{T_\delta}} values. In the controller of \cref{hapticFB} the torque gain is positive while in the control model of \cref{ch:4} it is negative and strangely close to one. Further insight into the importance of torque feedback for the steer perturbation experiment can be gained by performing a similar analysis as shown in \cref{hapticFB}. 
\chapter{Conclusions}
 
In an effort to iterate over existing rider control models, the VDROP model is created that successfully accounts for sensory delays by the use of an internal forward model. It is shown that implementation of delay without some compensation does not produce results that match the experimental data.  A prediction strategy is developed that manages to  circumvent the inability of the conventional Smith predictor to work on inherently unstable open loop systems by implementing a resetting forward model (DOP). The results matched the  measured non-parametric outputs with a good level of fit. Additionally, the model is found to be robust towards internal model inaccuracies. 
 
Furthermore, the importance of the sensory inflow from the Golgi tendon organs was thoroughly examined. From the results it is therefore concluded that a proper rider control model should include the torque feedback pathway. Contrary to what  the results of the analysis done by \citet{dialynaseffect} indicated, the results of \cref{hapticFB} suggest that torque feedback is in fact crucial to the execution of the balancing task. However the torque feedback in the experiments was not physiologically neutered and  state information could be deduced by the remaining inertial properties of the handlebar. Even though a steer-by-wire system decouples the roll and steer dynamics the remaining inertial feedback of the handlebar components (haptics off) was proven to be adequate for the rider to achieve comparable performance between conditions. Further experiments with negative stiffness applied at the handlebars could be conducted to cancel out inertial steering effects to experimentally validate  these results.
 
 
As far as the steering perturbation experiment is concerned, the rider model created (see \cref{ch:4}) manages to achieve a good level of fit and simultaneously captures the significance of the intrinsic response when countering steering torque perturbations. A high level of intersubject variability is exhibited. The hypothesis that this variability is in fact due to the modulation of admittance in the shoulder joint is strongly indicated. Despite all that,  the assumption that for constant speeds the system is time invariant does not seem to hold ground as noted by the irregular effect of cocontraction in the later stages of the rider  response.


\chapter*{Acknowledgements}

I would like to express my special thanks and gratitude to my close friend and supervisor  George Dialynas. Without his help I would still be trying to figure out the steer-by-wire bicycle, unable to acquire the dataset used for the analysis present in this thesis. Additionally, I would like to thank my colleagues in the Bicycle lab, specifically Sterre Kuipers and Koen Wendel, that made the routine of working every day in the lab such a pleasant experience. I would also like to thank Koen  for our countless discussions on rider control. Key insights from our conversations went into tackling various roadblocks in my thesis. Also I would like to thank Katerina Isaakidou a fellow Biomedical Engineering student who listened to my frustrations daily during lunchtime with the only thing expecting in return to listen to hers. Furthermore, a heartfelt thank you goes to our spanish visitor Ezequiel Martín Sosa who provided also his fair share of laughter and free coffee to the students of the lab. Finally I would like to thank my professor, Arend Schwab, for his supervision and guidance throughout my project.