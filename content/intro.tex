\chapter{Introduction}
\label{introduction}

Bicycle riding is a fundamental part of everyday transportation in many countries around the world. Ever since the development of the safety bicycle(two equal sized wheels, pneumatic tires, chain drive, rear wheel propulsion and a bent front fork), almost 130 years ago, the bicycle remains  one of the most prominent means of transport \cite{kooijman2013review}. With the growing concerns of sedentary lifestyles many choose the bicycle as their primary commute vehicle with the hopes of maintaining some levels of fitness. Additionally the bicycle is the preferred means of physical exercise for the elderly especially in the Netherlands and Denmark. Despite the fact that riding a bicycle is one of the first skills we acquire as kids and is used throughout the adult life, the fundamental way humans control the bicycle and ,generally single track vehicles,  is yet to be understood.\par
In a recent study examining the entries of patients to the emergency department due to traffic related accidents in the Netherlands (see Fig. \ref{fig:figure1}), it was found that bicycle related accidents were the most prevalent. With over 60,000 reported cases  bicycle accidents outnumber automobile accidents more than 4 to 1. It therefore  becomes clear that a lot could be done to improve cycling safety. Further look at the figure will reveal that most of those accidents did not involve a second party. There was  just a rider that fell off his bike. Although there are several potential reasons that riders lose control of the bicycle, formulating a general model of  how humans control single track vehicles could prove invaluable in understanding the causes behind the above numbers.

\begin{figure}[ht]
    \centering
    \includegraphics[scale=0.8]{images/figure1_1.png}
    \caption[Short title]{The number of road users that visit the emergency department at a hospital after a traffic accident in the Netherlands in 2016. Red indicates single vehicle accidents, yellow indicates a collision with an obstacle, blue indicates multi-vehicle accident and grey indicates other type of accidents\cite{krul_nijman_stam_2016}.}
    \label{fig:figure1}
\end{figure}

Every human-machine system requires an understanding of how the plant operates. In the case of the bicycle multiple efforts have been made in capturing the dynamics of the bicycle and its  self-stabilizing behavior. These have resulted in a set of linearized equations of motion, now  commonly referred to as the Whipple Carvalho model \cite{meijaard2007linearized}, which is going to be discussed further in chapter \ref{results}.\par
As it is known single track vehicles are not stable at low speeds. This is why a controller (like the human rider) is required to close the loop and create a stable system. There are two ways with which the human affects the dynamics of the plant. The first is with its passive interactions with the plant as a physical multibody system. Most passive models model the rider as a point mass rigidly attached to the rear frame, or as a pendulum connected to the rear frame \cite{eaton1975man}, although recent efforts have been made to extend this further with more complex passive rider models which include modeling of neuromuscular dynamics with spring-damper systems at various interfaces between rider and the bicycle frame\cite{dialynas2019}. The second is with its active control behavior. This involves the active control motion, such as steering, leaning or pedaling, applied by the rider to control and balance the bicycle. In most such cases, the passive behavior  of the rider is simplified by only  accounting for a fixed mass on the seat post, but when lean torque needs to be examined more complex modeling is required. The focus of the study is to explore the available models that best express the human rider as a controller in the bicycle-rider closed loop system.\par
The literature review presented herein gives an overview of research done in the field of active rider modelling  concerning single-track vehicles, while discussing with which methods and to what extent have these models been validated. This extensive overview is given in section \ref{sub:rider} of chapter \ref{results}, which is structured in three sections: \ref{ch:classical} Classical control system design, \ref{ch:optimal} Optimal control system design and \ref{ch:other} Other control system design. Chapter \ref{conclusions} concludes on the results from Chapter \ref{results} having as a final goal to answer the research question:
\par
• \textbf{What is the controller that best simulates the behavior of the human in the control of single track vehicles?}

